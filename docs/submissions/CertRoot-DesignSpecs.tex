\documentclass{article}



% Language setting

% Replace `english' with e.g. `spanish' to change the document language

\usepackage[english]{babel}



% Set page size and margins

% Replace `letterpaper' with `a4paper' for UK/EU standard size

\usepackage[letterpaper,top=2cm,bottom=2cm,left=3cm,right=3cm,marginparwidth=1.75cm]{geometry}



% Useful packages

\usepackage{amsmath}

\usepackage{graphicx}

\usepackage[colorlinks=true, allcolors=blue]{hyperref}



\title{CertRoot - Design Specifications}

\author{Nguyen, Thien Phuc}



\begin{document}

\maketitle

\section{Revision History}

\begin{center} 
\begin{tabular}{|l|l|l|l|}\hline
\textbf{Revision}& \textbf{Date}& \textbf{Created By}& \textbf{Description} \\\hline
0.1& October 1st, 2025& Nguyen, Thien Phuc&Initial version\\\hline
\end{tabular}
\end{center} 


\section{Project Overview}

\subsection{Problem}

In engineering and manufacturing, design reviews often end without an immutable record of the final files and feedback. This creates a trust gap, since firms cannot independently prove that an approved design hasn’t been altered. The result is exposure to IP risks, disputes, and supply chain liability.

\subsection{Solution}

CertRoot integrates with CoLab to create a blockchain-based audit log for every closed review. Final design files and comment threads are hashed and sealed on-chain, providing a verifiable record of integrity. This ensures designs can be independently validated at any time, giving firms stronger IP protection, reduced liability, and greater client confidence.

\subsection{Development team}

\begin{itemize}
\item Nguyen, Thien Phuc (Gerard)
\item Saha, Priyanka
\item Gajjar, Ronit Hirenbhai
\item Manyam, Anil
\item Alam, Sadia
\end{itemize}

\subsection{Client}

\begin{itemize}
\item Mr. Freddie Pike, Staff Developer and Technical Onboarding Manager at CoLab Software
\end{itemize}

\subsection{Project Timeline}

\begin{itemize}
\item Start date: September 30th, 2025
\item End date: November 30th, 2025 
\end{itemize}


\section{How CertRoot Works}

Think of a \textbf{hash} like a \textbf{fingerprint for a file}.

\begin{itemize}
\item If you change even one pixel in a drawing or one word in a comment, the fingerprint comes out completely different.
\end{itemize}

CertRoot takes the \textbf{fingerprint (hash) of your design files and reviews threads} and writes it to a \textbf{blockchain}.

\begin{itemize}
\item Blockchain is just a special type of database that nobody can secretly change. Once something is written, it stays there forever.
\end{itemize}

Later, if someone uploads the same file again, CertRoot re-creates the fingerprint and checks it against what’s stored on the blockchain.


\begin{itemize}
\item If they match → file is authentic, untouched since approval.
\item If they don’t match → file has been changed.
\end{itemize}

So in plain terms:


\begin{itemize}
\item \textbf{Cryptography = making a file’s unique fingerprint.}
\item \textbf{Blockchain = keeping that fingerprint in a tamper-proof ledger.}
\end{itemize}

Together, they give the 'digital notary' effect.


\section{Project Deliverables}

The final product is a modular suite of tools designed for seamless integration into CoLab’s Python backend and React frontend environments.


\subsection{\textbf{The Core Trust Layer (Solidity Smart Contract)}}

● \textbf{EVM Smart Contract:} The fully developed, tested, and audited Solidity smart contract deployed to the agreed-upon EVM-compatible network.

\begin{itemize}
    \item \textbf{Functions:} Contains only the essential public functions:
\end{itemize}

\begin{itemize}
    \begin{itemize}
        \item recordHash (for certification). 
        \item checkHashExists (for verification).
    \end{itemize}
\end{itemize}

\begin{itemize}
    \item \textbf{Immutability:} The final compiled ABI (Application Binary Interface) and Contract Address will be delivered for CoLab's integration reference.
\end{itemize}


\subsection{\textbf{The Python SDK (Backend Engine)}}

● \textbf{Python Package (certroot-sdk):} A fully documented, pip-installable Python package.

\begin{itemize}
    \item \textbf{Secure Certification:} 
\end{itemize}

\begin{itemize}
    \begin{itemize}
        \item The core function, sdk.certify\_file(file\_path, metadata), handles local hashing, secure wallet key management, gas estimation, and transaction signing via web3.py.
    \end{itemize}
\end{itemize}

\begin{itemize}

    \item \textbf{Integrity Check:} 
\end{itemize}

\begin{itemize}
    \begin{itemize}
        \item The sdk.check\_integrity(hash) function performs the simple query for internal system checks.
    \end{itemize}
\end{itemize}

\begin{itemize}

    \item \textbf{Code Quality:} 
\end{itemize}

\begin{itemize}
    \begin{itemize}
        \item All Python code will be unit-tested and conform to Python best practices, including robust exception handling for network and transaction failures.
    \end{itemize}
\end{itemize}


\begin{itemize}

    \item \textbf{Documentation:} 
\end{itemize}

\begin{itemize}
    \begin{itemize}
        \item Comprehensive README and integration guide with clear examples for CoLab engineers.
    \end{itemize}
\end{itemize}



\subsection{\textbf{The React Component Library (Frontend Integration)}}

● \textbf{NPM Package (@certroot/react-ui):} A modular npm package containing key components.


\begin{itemize}

    \item \textbf{Certification Component:} 
\end{itemize}


\begin{itemize}
    \begin{itemize}
        \item A drop-in \textless CertifyButton\textgreater  component that triggers the necessary API calls to the Python backend to initiate the certification process.
    \end{itemize}
\end{itemize}


\begin{itemize}

    \item \textbf{Verification Component:} 
\end{itemize}


\begin{itemize}
    \begin{itemize}
        \item A component like \textless IntegrityBadge\textgreater   that displays the verification status (e.g., "Certified" / "Tampered") directly in the CoLab UI.
    \end{itemize}
\end{itemize}


\begin{itemize}

    \item \textbf{Documentation:} 
\end{itemize}


\begin{itemize}
    \begin{itemize}
        \item Clear usage instructions, props definitions, and examples for quick adoption.
    \end{itemize}
\end{itemize}


\subsection{\textbf{The Public Verification Portal (External Assurance)}}

This crucial tool ensures third-party trust and compliance.


\begin{itemize}

    \item \textbf{Standalone Web Application:} 
\end{itemize}


\begin{itemize}
    \begin{itemize}
        \item A single-page application (SPA), hosted separately, designed for external use (auditors, clients).
    \end{itemize}
\end{itemize}


\begin{itemize}

    \item \textbf{Functionality:} 
\end{itemize}


\begin{itemize}
    \begin{itemize}
        \item Allows a user to upload a file to perform a local hash calculation and then makes a direct, free read-only view call to the EVM contract.
    \end{itemize}
\end{itemize}


\begin{itemize}

    \item \textbf{Trust:} 
\end{itemize}


\begin{itemize}
    \begin{itemize}
        \item Provides independent, trustless proof of integrity without requiring the auditor to interact with CoLab's production systems.
    \end{itemize}
\end{itemize}




\section{Functional Requirements}

\subsection{\textbf{Python SDK}}

\begin{center}
\begin{tabular}{|l|p{10cm}|}\hline 
  \textbf{Requirement}&\textbf{Description}\\\hline
  Local File Hashing&
    \begin{itemize}
        \item Must generate SHA-256 hash values for files and comment threads locally on server. 
        \item supporting large CAD/image/PDF files via efficient I/O streaming.
    \end{itemize}\\\hline 

  Secure EVM Certification&
    \begin{itemize}
        \item sdk.certify\_file(), must securely manage the certification wallet's private key.
    \end{itemize}\\\hline 


  Record Data Structure&
  Input:
    \begin{itemize}
        \item The file hash(es). 
        \item Comment hash.
        \item CoLab metadata (User ID, Project ID).
        \item Timestamp.
    \end{itemize}
    Output:
    \begin{itemize}
        \item A single, compact data structure for the Solidity contract.
    \end{itemize}\\\hline 


  Internal Integrity Check&
    \begin{itemize}
        \item sdk.check\_integrity(hash) method to quickly query the EVM contract for a hash's existence.
    \end{itemize}\\\hline 


  CoLab Workflow Integration&
    \begin{itemize}
        \item Provide clear hooks to automatically trigger certification upon a 'ReviewClosed' event in CoLab's Python backend.
    \end{itemize}\\\hline 
\end{tabular}
\end{center}

\subsection{\textbf{EVM Smart Contract}}

\begin{center}
\begin{tabular}{|l|p{10cm}|}\hline 
  \textbf{Requirement}&\textbf{Description}\\\hline
  Data Immutability&
    \begin{itemize}
        \item Store hash records permanently on the EVM-compatible chain using the recordHash() function.
    \end{itemize}\\\hline 

  Verification Retrieval&
    \begin{itemize}
        \item Expose a read-only view function (checkHashExists()) that allows for quick, gas-free public verification checks.
    \end{itemize}\\\hline 


  ABI and Address&
  \begin{itemize}
        \item The final Contract Address and the Application Binary Interface (ABI) JSON file must be delivered for all integration efforts.
    \end{itemize}\\\hline 
\end{tabular}
\end{center}



\subsection{\textbf{React Component Library \& Portal }}

\begin{center}
\begin{tabular}{|l|p{10cm}|}\hline 
  \textbf{Requirement}&\textbf{Description}\\\hline
  Certification Trigger Component&
    \begin{itemize}
        \item Deliver a reusable React component (e.g., \textless CertifyButton\textgreater  ) that triggers the certification flow via CoLab's backend API.
    \end{itemize}\\\hline 

  Public Verification Portal&
    \begin{itemize}
        \item A standalone React application where an external user can upload a file, calculate the hash locally in the browser, and make a direct EVM view call for verification.
    \end{itemize}\\\hline 


  Display Results&
  Verification results: 
  \begin{itemize}
\item Must be clear and display block details.
\item Show Timestamp. 
\item Show metadata. 
\item URL link to the EVM block explorer.
    \end{itemize}\\\hline 
\end{tabular}
\end{center}

\section{Non-Functional Requirements}

\begin{center}
\begin{tabular}{|l|p{10cm}|}\hline 
  \textbf{Requirement}&\textbf{Description}\\\hline
  Security (Wallet)&
    \begin{itemize}
        \item Absolute priority must be placed on the secure management and isolation of the Python SDK's private key used for signing certification transactions.
    \end{itemize}\\\hline 

  Performance (Hashing)&
    \begin{itemize}
        \item Python SDK hashing implementation must be optimized to handle multi-gigabyte files efficiently to avoid backend latency.
    \end{itemize}\\\hline 
  Usability (React)&
    \begin{itemize}
\item The React components and the Verification Portal must adhere to CoLab's UI standards. 
\item Provide clear status feedback (e.g., "Transaction Pending," "Certified") to minimize user confusion about blockchain processes.
    \end{itemize}\\\hline
 Scalability (EVM)&
    \begin{itemize}
        \item The smart contract must be gas-optimized to ensure transaction costs remain low even as the usage volume increases.
    \end{itemize}\\\hline
 Documentation&
    \begin{itemize}
        \item Provide comprehensive documentation for the Python SDK (installation, usage) and the React components (props, examples).
    \end{itemize}\\\hline 
\end{tabular}
\end{center}


\section{Use Cases}

\begin{center}
\begin{tabular}{|p{2cm}|p{3cm}|p{8.5cm}|}\hline
\textbf{Use Case} & \textbf{Primary Actor/Component} & \textbf{Description (EVM \& SDK Context)}\\\hline
\multicolumn{3}{|l|}{\textbf{1. Close Review \& Seal Design}}\\\hline
 & CoLab Backend (Python SDK) &
\begin{itemize}
    \item The Python SDK automatically triggers upon the ReviewClosed event.
    \item It securely hashes all final artifacts, bundles the data
    \item It uses the certification wallet to sign and submit the transaction (paying gas) to the EVM contract.
\end{itemize}\\\hline

\multicolumn{3}{|l|}{\textbf{2. Verify File Integrity}}\\\hline
 & External Auditor (Public React Portal) &
\begin{itemize}
    \item The auditor uploads the file to the React Portal.
    \item The browser calculates the hash and makes a direct, gas-free view call.
    \item EVM contract is deployed for instant, trustless verification.
\end{itemize}\\\hline

\multicolumn{3}{|l|}{\textbf{3. API-based Integration}}\\\hline
 & CoLab Backend Engineering Team &
\begin{itemize}
    \item A user imports the certroot-sdk Python package.
    \item \texttt{sdk.certify\_file()} can be integrated into core review closure logic, abstracting the complex EVM process.
\end{itemize}\\\hline

\multicolumn{3}{|l|}{\textbf{4. View Blockchain Record}}\\\hline
 & CoLab User (React Component) &
\begin{itemize}
    \item A user clicks an "Integrity Proof" link.
    \item The React Component retrieves the transaction details from a public EVM Block Explorer.
    \item The system will redirect and display the timestamp and certification metadata.
\end{itemize}\\\hline

\multicolumn{3}{|l|}{\textbf{5. Optional File Archival}}\\\hline
 & Python SDK \& External Storage &
\begin{itemize}
    \item The Python SDK manages the secure upload of large files to an external system (e.g., S3).
    \item It then logs the file's hash and the resulting storage URL/pointer in the EVM contract metadata.
\end{itemize}\\\hline

\multicolumn{3}{|l|}{\textbf{6. Audit Trail \& Reporting}}\\\hline
 & Compliance Officer (CoLab Backend Logic) &
\begin{itemize}
    \item Python SDK can be used to query a range of recorded hashes from the EVM contract.
    \item Provide a compiled report that shows every sealed design for a given project ID for regulatory review.
\end{itemize}\\\hline

\end{tabular}
\end{center}


\section{Development Schedule}

\subsection{\textbf{Phase 1: Foundation \& Backend Focus}}

\begin{center}
\begin{tabular}{|p{2.5cm}|p{5cm}|p{4cm}|p{2.5cm}|}\hline
\textbf{Date Range} & \textbf{Focus (4 Devs)} & \textbf{Solo Focus (1 Dev)} & \textbf{Client Meeting} \\\hline
% Row 1
Oct 1 – Oct 4 & 
    Project setup: EVM/Solidity environment, Python SDK boilerplate, initial CI/CD foundation. & 
    Verification Portal UI: Build the front-end shell for the public verification tool (React). & 
    None \\\hline
% Row 2
Oct 7 – Oct 11 & 
    Solidity MVP: Develop the immutable smart contract (recordHash). Develop the core Python hashing function. & 
    Verification Portal UX: Design and implement the verification result display logic. & 
    Oct 13: Kick-off/Review \\\hline
% Row 3
Oct 14 – Oct 18 & 
    Python SDK Write Logic: Implement web3.py for secure transaction signing (wallet management, gas estimation). & 
    Demo Prep: Integrate the client-side hashing function for the public portal. & 
    None \\\hline
% Row 4
Oct 21 – Oct 25 & 
    Integration \& Test: Full E2E testing (Python to Solidity testnet). Finalize Python SDK CI/CD. & 
    Demo Prep: Final E2E testing of the verification read function on the portal. & 
    Oct 25: Phase 1 Demo \\\hline
\end{tabular}
\end{center}


\subsection{\textbf{Phase 2: Client Integration \& Polish }}

\begin{center}
\begin{tabular}{|p{2.5cm}|p{5cm}|p{4cm}|p{2.5cm}|}\hline
\textbf{Date Range} & \textbf{Focus (4 Devs)} & \textbf{Maintenance (1 Dev)} & \textbf{Client Meeting} \\\hline
% Row 1
Nov 4 – Nov 8 & 
    React Component 1: Develop the core <CertifyButton> and set up the NPM package structure and CI/CD. & 
    Maintenance \& Docs: Write comprehensive Python SDK documentation and fix any Phase 1 issues. & 
    None \\\hline
% Row 2
Nov 11 – Nov 15 & 
    React Component 2 \& Portal: Develop the <IntegrityBadge> and finalize the Public Verification Portal UI/UX. & 
    Refinement: Final Python SDK changes based on client review feedback. & 
    Nov 17: Component Review \\\hline
% Row 3
Nov 18 – Nov 22 & 
    QA \& Packaging: Full integration testing (React to Python SDK). Final NPM package and Python wheel preparation. & 
    Release Prep: Finalize all documentation, release notes, and prepare for handover. & 
    None \\\hline
% Row 4
Nov 25 – Nov 29 & 
    Final Submission: Prepare the final presentation and deliver all source code, libraries, and documentation. & 
    Handover Prep: Prepare a technical walkthrough of the Python SDK security features. & 
    Dec 1: Final Handover \\\hline
\end{tabular}
\end{center}

\end{document}